\documentclass[12pt]{article}
\begin{document}
\section*{Background}
\textit{Introduce the research topic. Place the project in academic or
professional context by referring to major works by others on the
subject. }

Understanding the accelerated expansion of the is one of the most
important unsolved problems in fundamental physics.  
This was recognized by the Nobel Prize on Physics to three astronomers
who first found observational evidence for this phenomeno.
A complete understanding of the accelerated lies either on a new kind
of negative pressure energy component (dubbed under the generic name
of Dark Energy) or in the modification of Einstein's General
Relativity. 

The Dark Energy Spectroscopic Instrument (DESI) is a world-class
experiment that is designed to bring the most significant experimental
advance on this front. 
DESI will take the spectra of 35 million galaxies to make most
accurate measurement of the expansion history of the Universe on the
timescale 2019-2024.  

One of the significant milestones to prepare DESI operations is a
full simulation of the experiment.  
This simulation will allows to stress-test all the software that will
be used on the real data and also identify possible strategies to
maximize DESI's scientific return.  

The work to simulate such a complex experiment integrates the expertise of
different groups of experts. From the engineers designing 
the instrument, to the astronomers processing the data, going between
all the expert simulators of universes, galaxies and stars. 




\section*{Objectives}
% Objectives: 
\textit{Clearly define the aims of the project.}

The main objetive of this project is to analyze the simulation of five
years DESI operations.

This will allow us to reach three goals:
\begin{itemize}
\item Forecast the accuracy at which the Dark Energy Spectroscopic
  Instrument will be able to constraint the expansion history of the Universe. 
\item Quantify the degree to which different instrumental systematic
  errors can degrade the experiment's performance.
\item Test strategies to mitigate systematic effects and maximize
  DESI's scientific return.
\end{itemize}
\section*{Methodology}
%Methodology: Describe the project. Explain the approach, methods and
%plan you will use (for example, interviews, library or archival
%research, or laboratory experiments). Indicate whether the proposed
%research is quantitative or qualitative.  
Perform simulations 

\section*{Significance}
%Significance: Explain the importance of the project for the field,
%your home country and your own professional development. Indicate
%what effect you expect the opportunity to have on your teaching or
%professional work in your home country. (For example: new approaches
%to curriculum planning, student advising or pedagogy; expanding
%knowledge in the field through collaboration with
%U.S. colleagues). Describe briefly the expected impact of your
%participation on your home institution, community or professional
%field. 

Importance to the international community.

Importance to the national community.

Importance to my research.

In my teaching.

In Astronomy for Development.

\section*{Evaluation and Dissemination}
\textit{Evaluation and Dissemination: Describe plans for assessment and
distribution of research results in your home country and elsewhere.}

Contribution to software.

Publications by the collaboration.

Assesment by the collaboration leaders. 

Presentation in the collaboratio meeting.

Presentations in the Colombian Congress of Astronomy. 



\section*{Justification}
\textit{Justification for Residence in the United States for the Proposed
Project: Indicate why it is necessary to conduct the research onsite
 in the United States. }

DESI is coordinated by the Lawrence Berkeley National Laboratory.
Most of the simulation work is done in different locations around the
world (incluing Bogota) and it is coordinated via virtual meetings. 
However, to consolidate progress (i.e. by releasing software or data)
it is crucial to spend a minimal amount of time in face-to-face meetings.

Since 2014 I have made great efforts to spend one week per semester at
BerkeleyLab to efficiently contribute my effort to the collaboration. 
As DESI gets closer to start operations, the experiment has reached a
point where the end-to-end simulation effort has matured and gained
relevance to the operationl aspects of the project.

A significant contribution at this time needs the focused effort on
site that a Fulbright fellowship can provide. 

\section*{Duration}
\textit{ Duration: Explain how the project can be completed within the time
period proposed. }

I expect to spend a total of 16 weeks working on site at Berkeley
lab. 

\end{document}
