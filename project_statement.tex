\documentclass[12pt]{article}
\title{Pushing the Limits of Dark Energy Experiments}
\begin{document}
\maketitle
\section*{Background}
\textit{Introduce the research topic. Place the project in academic or
professional context by referring to major works by others on the
subject. }

Understanding the accelerated expansion of the Universe is one of the
most important unsolved problems in fundamental physics.  
This was recognized by the 2011 Nobel Prize on Physics to three
astronomers who lead research teams that found observational evidence for
this phenomenon.  A complete understanding of the accelerated
expansion lies either on a new kind of negative pressure energy
component (dubbed under the generic name of Dark Energy) or in the
modification of Einstein's General Relativity. 

The Dark Energy Spectroscopic Instrument (DESI) is a world-class
experiment that is designed to bring the most significant experimental
advance on this front. 
DESI will take the spectra of 35 million galaxies to make most
accurate measurement of the expansion history of the Universe on the
timescale 2019-2024 using the Baryon Acoustic Oscillation technique.  

One of the significant milestones to prepare DESI operations is a
full simulation of its expected five years of operations.
This simulation will allow to stress-test all the software that will
be used on the real data and also identify in advance strategies to
maximize DESI's scientific return.  
The results will be useful for the broad scientific community
expecting to use DESI data products.

The work to simulate such a complex experiment integrates the knowledge of
different groups of experts \cite{2016A&C....15....1N}. From the
engineers designing   the instrument, to the astronomers processing
the data, including the expert simulators of universes and
galaxies. This is a process I have been involved with during the last
2 years.
After the simulation is completed by mid 2017, I will join the effort
to analyze the simulated results of the simulated DESI experiment.

 
This work will allows us to understand how close we are to answering
one of the most fundamentals questions about Science and Nature. What
will be the power of DESI to decide whether Einstein's General
Relativity is correct?  
Perhaps more exciting is that this would be our first step into the
unknown with DESI. What do we need to do to take the experiment beyond its
original design limits? Where are exactly DESI's limits for discovery?

\bibliography{references}{}
\bibliographystyle{plain}




\section*{Objectives}
% Objectives: 
\textit{Clearly define the aims of the project.}

The main objective is to analyze the data resulting from a simulation
of five years of DESI operations.  

This will allow us to reach three goals:
\begin{itemize}
\item Forecast with great detail the accuracy at which the Dark Energy
  Spectroscopic Instrument will be able to constraint the expansion
  history of the Universe.  
\item Quantify the degree to which different instrumental systematic
  errors can degrade the experiment's performance.
\item Test strategies to mitigate systematic effects and maximize
  DESI's scientific return.
\end{itemize}
\section*{Methodology}
\textit{Methodology: Describe the project. Explain the approach, methods and
plan you will use (for example, interviews, library or archival
research, or laboratory experiments). Indicate whether the proposed
research is quantitative or qualitative. }



\section*{Significance}
\textit{Significance: Explain the importance of the project for the field,
your home country and your own professional development. Indicate
what effect you expect the opportunity to have on your teaching or
professional work in your home country. (For example: new approaches
to curriculum planning, student advising or pedagogy; expanding
knowledge in the field through collaboration with
U.S. colleagues). Describe briefly the expected impact of your
participation on your home institution, community or professional
field. }


Importance to the international community.

Importance to the Colombian community.

Importance to my research.

Impact on my teaching.

Importance to the Astronomy for Development community.

\section*{Evaluation and Dissemination}
\textit{Evaluation and Dissemination: Describe plans for assessment and
distribution of research results in your home country and elsewhere.}

Contribution to software.

Publications by the collaboration.

Assessment by the collaboration leaders. 

Presentation in the collaboration meeting.

Presentations in the Colombian Congress of Astronomy. 



\section*{Justification}
\textit{Justification for Residence in the United States for the Proposed
Project: Indicate why it is necessary to conduct the research on site
 in the United States. }

DESI is coordinated by the Lawrence Berkeley National Laboratory. The
collaboration includes close to 300 scientists in 40 institutions
around the world. 
Most of the simulation work is done in different locations (including
Bogota) and it is coordinated via virtual meetings.  
However, to consolidate progress (i.e. by releasing software or data
to the collaboration) it is crucial to spend a minimal amount of time
in face-to-face meetings. 

Since 2014 I have made great efforts to spend one week per semester at
Berkeley Lab to efficiently contribute my effort to the collaboration. 
As DESI gets closer to starting operations in January 2019, the
experiment has reached a point where the end-to-end simulation effort
has matured and gained relevance to the operational aspects of the
project. 

A significant and timely contribution to DESI needs the focused effort on
site at Berkeley Lab that only a Fulbright fellowship can provide. 

\section*{Duration}
\textit{ Duration: Explain how the project can be completed within the time
period proposed. }

I expect to spend a total of 16 weeks working on site at Berkeley
Lab starting mid August 2017 through December 2017. 

Currently we are working on the preparation of the full simulation
(August-December 2016) and the associated analysis tools. We will
perform the simulation next year (January-July 2017) and complete the
analysis in the period August-December 2017.  

The analysis stage will be the focus of my visit
to Berkeley Lab.
We foresee four stages, each one fitting into a 4-week period.

\begin{enumerate}
\item Measure the accuracy at which the simulated data constrains the
  expansion history of the Universe.  
\item Quantify different instrumental systematic errors can degrade
  the experiment's performance. The focus will be on the effect of
  fiber assignment and success of the spectroscopic pipeline. 
\item Run simplified simulations to test strategies that mitigate
  systematic effects and maximize DESI's scientific return.
\item Consolidate a detailed report on the results from the end-to-end
  simulation challenge. 
  The full report is aimed at the DESI collaboration. A condensed
  version will be published for the broader scientific community
  interested in understanding the kind of data that the DESI
  collaboration will produce. 
\end{enumerate}

This tasks depend on software that we are constrained to develop in
the period August 2016 - July 2017. As the simulation runs we will
analyze small chunks of data to fine tune the required tools. 

The experience we gain in this one-year preparation period ensures that
focused work during 16 weeks, using mostly existing tools, is the only
critical step to succeed in the global analysis task.

\end{document}
