\documentclass[12pt]{article}
\begin{document}
\section*{Background}
% Introduce the research topic. Place the project in academic or
% professional context by referring to major works by others on the
% subject. 
A
\section*{Objectives}
% Objectives: Clearly define the aims of the project. 
Forecast the accuracy at which the Dark Energy Spectroscopic
Instrument will be able to constraint the expansion history of the
Universe. 

\section*{Methodology}
%Methodology: Describe the project. Explain the approach, methods and
%plan you will use (for example, interviews, library or archival
%research, or laboratory experiments). Indicate whether the proposed
%research is quantitative or qualitative.  
Perform simulations 

\section*{Significance}
%Significance: Explain the importance of the project for the field,
%your home country and your own professional development. Indicate
%what effect you expect the opportunity to have on your teaching or
%professional work in your home country. (For example: new approaches
%to curriculum planning, student advising or pedagogy; expanding
%knowledge in the field through collaboration with
%U.S. colleagues). Describe briefly the expected impact of your
%participation on your home institution, community or professional
%field. 

DESI is a world-class experiment.

\section*{Evaluation and Dissemination}
%Evaluation and Dissemination: Describe plans for assessment and
%distribution of research results in your home country and elsewhere. 

Presentations in the Colombian Congress of Astronomy. 


\section*{Justification}
% Justification for Residence in the United States for the Proposed
% Project: Indicate why it is necessary to conduct the research onsite
% in the United States. 
DESI is conducted by the Lawrence Berkeley National Laboratory.

\section*{Duration}
%Duration: Explain how the project can be completed within the time
%period proposed. 
A

\end{document}
