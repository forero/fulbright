\documentclass[12pt]{article}
\setlength{\oddsidemargin}{0cm}
\setlength{\textwidth}{440pt}
\setlength{\topmargin}{-40pt}
\addtolength{\hoffset}{+1.0cm}
\addtolength{\textheight}{4cm}
\usepackage{hyperref}
\usepackage{fancyhdr}
\usepackage{apacite}
%\fancyhead[RE,LO]{Message of the day} 
\fancyhead[L]{Jaime E. Forero-Romero, Colombia, Pushing the Limits of
  Dark Energy Experiments} 
\title{Pushing the Limits of Dark Energy Experiments}
\author{Jaime E. Forero-Romero\\Colombia}
\begin{document}
\maketitle
\pagestyle{empty}
\pagestyle{fancy}
\section*{Background}
%\textit{Introduce the research topic. Place the project in academic or
%professional context by referring to major works by others on the
%subject. }

Understanding the accelerated expansion of the Universe is one the
biggest unsolved problems in fundamental physics.  
This was recognized by the 2011 Physics Nobel Prize to the three
astronomers who lead the research teams that found observational evidence for
this phenomenon \cite{1998AJ....116.1009R,1999ApJ...517..565P}.  
The main reason for its protagonism in fundamental physics is that a
complete understanding of the accelerated expansion lies either in a
new kind of negative pressure energy component (dubbed under the
generic name of Dark Energy) or in the modification of Einstein's
General Relativity.   

The Dark Energy Spectroscopic Instrument (DESI) is a world-class
experiment that is designed to bring the most significant experimental
advance on this front. 
DESI will take the spectra of 35 million galaxies to make the most
accurate measurement of the expansion history of the Universe on the
timescale 2019-2024 using the Baryon Acoustic Oscillation (BAO) technique
\cite{2005ApJ...633..560E,2005MNRAS.362..505C}.    
DESI will be the leading experiment in observational cosmology during
that period.  

One of the significant milestones to prepare DESI is a full simulation
of its expected five years of operations. 
This simulation will test all the software components that will be used on
the real data. It will also help to test strategies to  maximize
DESI's scientific return.   
The results of this simulation exercise will also be useful to the
broad scientific community expecting to use DESI data products. 
 
The work to simulate DESI builds upon the accumulated experience of
its precursors \cite{2013AJ....145...10D}, it also integrates the knowledge of
different groups of experts \cite{2016A&C....15....1N}. From the
engineers designing  the instrument, to the astronomers processing the
data, including the expert simulators of universes and galaxies. 
In this project we will work with data from this simulation effort to
help DESI in getting closer to answer fundamental questions about
the nature of our Universe.  

%I have been involved in the full simulation process during the last two
%years. I have been writing software and curating data.  
%The After the
%simulation is completed by mid 2017, I will join the effort %
%to analyze the resulting data.
 


%answering one of the most fundamentals questions about Science and
%Nature. What will be the power of DESI to decide whether Einstein's
%General Relativity is correct?  
%Perhaps more exciting is that this would be our first step into the
%unknown with DESI. What do we need to do to take the experiment beyond its
%original design limits? Where are exactly DESI's power and limits for
%discovery?  


\section*{Objectives}
% Objectives: 
%\textit{Clearly define the aims of the project.}

The main objective is to analyze the data resulting from a simulation
of five years of DESI operations.  
This will allow us to reach three goals:

\begin{itemize}
\item Forecast DESI's accuracy to constraint the expansion
  history of the Universe.  
\item Quantify the degree to which different instrumental systematic
  errors can degrade DESI's performance.
\item Test strategies to mitigate systematic effects and maximize
  DESI's scientific return.
\end{itemize}


\section*{Methodology}
%\textit{Methodology: Describe the project. Explain the approach, methods and
%plan you will use (for example, interviews, library or archival
%research, or laboratory experiments). Indicate whether the proposed
%research is quantitative or qualitative. }


This project aims at quantifying the expected performance of the DESI
experiment. 
This requires an end-to-end simulation of the DESI survey including:
mock catalogs for the galaxy populations to be
  observed \cite{2015AAS...22533607S}, simulating the 5 years of
  telescope operations including weather and observational cadence, assigning optical fibers to the
  galaxy targets \cite{2015AAS...22533610C}, generating simulations
  of the raw spectrograph data \cite{2015AAS...22533608E} and
  processing those simulated data with the full spectroscopic pipeline, resulting in a redshift catalog. Every link n this chain
  contributes its share of inefficiencies to the final measurement of
  the BAO signal \cite{2015AAS...22533605E}.

The simulation process allows the DESI project to test dataflows, check software
scaling  and generate realistic datasets for science preparation before
the survey starts. The current project proposal takes a starting point the
realistic simulated datasets for science preparation. Using this data we will
use and develop software to  
\begin{itemize}
\item Estimate in detail the accuracy at which DESI can measure the
  BAO signal at different cosmic epochs.
\item Quantify the impact on the BAO measurement inefficiencies from
  major parts/steps in the survey. 
\item Implement strategies to reduce the above mentioned inefficiencies. 
\end{itemize}

The main methodology thus consists in {\bf using software to make a
simulation-based quantitative assessments about DESI's performance in
measuring the BAO signal}. 

\section*{Significance}
%\textit{Significance: Explain the importance of the project for the field,
%your home country and your own professional development. Indicate
%what effect you expect the opportunity to have on your teaching or
%professional work in your home country. (For example: new approaches
%to curriculum planning, student advising or pedagogy; expanding
%knowledge in the field through collaboration with
%U.S. s colleagues). Describe briefly the expected impact of your
%articipation on your home institution, community or professional
%field. }

%Importance to the international community.
First and foremost, the significance of this project lies in its
potential to expand our knowledge about the physical Universe. DESI is
a world-class experiment in the area of observational cosmology and
will be second to none in this research field during its operational
time.  
 

%Importance to the Colombian community.
This collaboration will also enrich the Colombian scientific
community. Colombia participates in very few world class experiments
in the physical sciences. Being able to contribute to DESI will have a
long lasting impact in the new generation of young scientist involved
in the project as students.  

This will also naturally impact my career. Participation in DESI
allows me to contribute my research skills at the service of a large
collaboration, increasing the chances to tackle  projects of
ever increasing complexity and relevance to the scientific community.  
%Importance to my research.

%Impact on my teaching.

%Importance to the Astronomy for Development community.

I also work as a regional coordinator for the \emph{Astronomy for
Development} (AfD)
initiative \footnote{\url{http://andean.astro4dev.org/}}, whose motto
is to use astronomy to create a better world. The AfD initiative
deploys projects that use the cultural, technical  and scientific
aspects of astronomy to engage with local communities and impact their
development status.  My time at the Lawrence Berkeley National
Laboratory (the host institution) will allow me to create new
partnerships and gather novel ideas to achieve the AfD goals.  



\section*{Evaluation and Dissemination}
%\textit{Evaluation and Dissemination: Describe plans for assessment and%istribution of research results in your home country and elsewhere.}


%Assessment by the collaboration leaders. 
DESI has a formal project management structure. 
Under this framework there are well defined milestones
and deadlines for the different tasks defined in the current
proposal. The internal evaluation process by the managers in the
collaboration provides a natural strategy to asses the project's success.  

\noindent
I also foresee three main strategies to disseminate the results of this
work.
% osfware contribution
\begin{enumerate}
\item The software contributions written during the project's
development will all be publicly available through the public DESI
repository \footnote{\url{https://github.com/desihub/}}.  


%Publications by the collaboration.
\item The analysis of the data challenge results are of interest to
the scientific community. As such we will publish our conclusions
in astronomical journals of wide international circulation. 

\item The project results will also be presented in scientific
meetings both international and local (Colombian/Latinamerican) 
to enhance the impact on the development of the Colombian astronomical
community,  
\end{enumerate}



\section*{Justification}
%\textit{Justification for Residence in the United States for the Proposed
%Project: Indicate why it is necessary to conduct the research on site
% in the United States. }

DESI is coordinated by the Lawrence Berkeley National Laboratory. The
collaboration includes close to 300 scientists in 40 institutions
around the world. 
Most of the simulation work is done in different locations (including
Bogota) and it is coordinated via virtual meetings.  
However, to consolidate progress (i.e. by releasing software or data
to the collaboration) it is crucial to spend a minimal amount of time
in face-to-face meetings. 

Since 2014 I have used funding from my University and the DESI
collaboration to spend at least one week per semester at
Berkeley Lab writing and integrating software for the data simulation
pipeline. 
As DESI gets closer to starting operations in January 2019, the
experiment has reached a point where the end-to-end simulation effort
has matured and gained relevance to the operational aspects of the
project. 

A significant and timely contribution to DESI needs the focused effort on
site at Berkeley Lab that a Fulbright fellowship can best provide. 

\section*{Duration}
%\textit{ Duration: Explain how the project can be completed within the time
%period proposed. }

I expect to spend a total of 16 weeks working on site at Berkeley
Lab starting mid August 2017 through December 2017. 

Currently we are working on the preparation of the full simulation
(August-December 2016) and the associated analysis tools. We will
perform the simulation next year (January-July 2017) and complete the
analysis in the period August-December 2017.  

The analysis stage will be the focus of my visit
to Berkeley Lab.
We foresee four stages, each one fitting into a 4-week period.

\begin{enumerate}
\item Measure the accuracy at which the simulated data constrains the
  expansion history of the Universe.  
\item Quantify different instrumental systematic errors can degrade
  the experiment's performance. The focus will be on the effect of
  fiber assignment and success of the spectroscopic pipeline. 
\item Run simplified simulations to test strategies that mitigate
  systematic effects and maximize DESI's scientific return.
\item Consolidate a detailed report on the results from the end-to-end
  simulation challenge. 
  The full report is aimed at the DESI collaboration. A condensed
  version will be published for the broader scientific community
  interested in understanding the kind of data that the DESI
  collaboration will produce. 
\end{enumerate}

These tasks depend on software that we are constrained to develop in
the period August 2016 - July 2017. As the simulation progresses we
will analyze small chunks of data to fine tune the required tools.  

The experience we gain in this one-year preparation period ensures that
being able to do focused work during 16 weeks on site at LBL, using
mostly existing tools, is the only critical step to succeed in the
global analysis task. 

\newpage

\bibliographystyle{apacite}
%\bibliographystyle{ieeetr}
\bibliography{references}{}


\end{document}
